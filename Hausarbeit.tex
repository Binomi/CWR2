% Für Bindekorrektur als optionales Argument "BCORfaktormitmaßeinheit", dann
% sieht auch Option "twoside" vernünftig aus
% Näheres zu "scrartcl" bzw. "scrreprt" und "scrbook" siehe KOMA-Skript Doku
\documentclass[12pt,a4paper,titlepage,headinclude,bibtotoc]{scrartcl}


%---- Allgemeine Layout Einstellungen ------------------------------------------

% Für Kopf und Fußzeilen, siehe auch KOMA-Skript Doku
\usepackage[komastyle]{scrpage2}
\pagestyle{scrheadings}
\automark[section]{chapter}
\setheadsepline{0.5pt}[\color{black}]

%keine Einrückung
\parindent0pt

%Einstellungen für Figuren- und Tabellenbeschriftungen
\setkomafont{captionlabel}{\sffamily\bfseries}
\setcapindent{0em}


%---- Weitere Pakete -----------------------------------------------------------
% Die Pakete sind alle in der TeX Live Distribution enthalten. Wichtige Adressen
% www.ctan.org, www.dante.de

% Sprachunterstützung
\usepackage[ngerman]{babel}

% Benutzung von Umlauten direkt im Text
% entweder "latin1" oder "utf8"
\usepackage[utf8]{inputenc}

% Pakete mit Mathesymbolen und zur Beseitigung von Schwächen der Mathe-Umgebung
\usepackage{latexsym,exscale,amssymb,amsmath}

% Weitere Symbole
\usepackage[nointegrals]{wasysym}
\usepackage{eurosym}

% Anderes Literaturverzeichnisformat
%\usepackage[square,sort&compress]{natbib}

% Für Farbe
\usepackage{color}

% Zur Graphikausgabe
%Beipiel: \includegraphics[width=\textwidth]{grafik.png}
\usepackage{graphicx}

% Text umfließt Graphiken und Tabellen
% Beispiel:
% \begin{wrapfigure}[Zeilenanzahl]{"l" oder "r"}{breite}
%   \centering
%   \includegraphics[width=...]{grafik}
%   \caption{Beschriftung} 
%   \label{fig:grafik}
% \end{wrapfigure}
\usepackage{wrapfig}

% Mehrere Abbildungen nebeneinander
% Beispiel:
% \begin{figure}[htb]
%   \centering
%   \subfigure[Beschriftung 1\label{fig:label1}]
%   {\includegraphics[width=0.49\textwidth]{grafik1}}
%   \hfill
%   \subfigure[Beschriftung 2\label{fig:label2}]
%   {\includegraphics[width=0.49\textwidth]{grafik2}}
%   \caption{Beschriftung allgemein}
%   \label{fig:label-gesamt}
% \end{figure}
\usepackage{subfigure}

% Caption neben Abbildung
% Beispiel:
% \sidecaptionvpos{figure}{"c" oder "t" oder "b"}
% \begin{SCfigure}[rel. Breite (normalerweise = 1)][hbt]
%   \centering
%   \includegraphics[width=0.5\textwidth]{grafik.png}
%   \caption{Beschreibung}
%   \label{fig:}
% \end{SCfigure}
\usepackage{sidecap}

% Befehl für "Entspricht"-Zeichen
\newcommand{\corresponds}{\ensuremath{\mathrel{\widehat{=}}}}

%Für chemische Formeln (von www.dante.de)
%% Anpassung an LaTeX(2e) von Bernd Raichle
\makeatletter
\DeclareRobustCommand{\chemical}[1]{%
  {\(\m@th
   \edef\resetfontdimens{\noexpand\)%
       \fontdimen16\textfont2=\the\fontdimen16\textfont2
       \fontdimen17\textfont2=\the\fontdimen17\textfont2\relax}%
   \fontdimen16\textfont2=2.7pt \fontdimen17\textfont2=2.7pt
   \mathrm{#1}%
   \resetfontdimens}}
\makeatother

%Si Einheiten
\usepackage{siunitx}

%c++ Code einbinden
\usepackage{listings}
\lstset{numbers=left, numberstyle=\tiny, numbersep=5pt}

%Differential
\newcommand{\dif}{\ensuremath{\mathrm{d}}}

%Boxen,etc.
\usepackage{fancybox}
\usepackage{empheq}

%Fußnoten auf gleiche Seite
\interfootnotelinepenalty=1000

\begin{document}

\begin{titlepage}
\centering
\textsc{\Large CWR 2 SoSe 2014, Fakultät für
  Physik,\\[1.5ex] Universität Göttingen}

\vspace*{4.2cm}

\rule{\textwidth}{1pt}\\[0.5cm]
{\huge \bfseries
  Projekt 65\\[1.5ex]
  Doppelpendel}\\[0.5cm]
\rule{\textwidth}{1pt}

\vspace*{3.0cm}

\begin{Large}
\begin{tabular}{lr}
 Bearbeiter:  &  Felix Kurtz\\
 E-Mail: &  felix.kurtz@stud.uni-goettingen.de\\
 Betreuer: & Burkhard Blobel \\
 Abgabe: & 18.08.2014\\
\end{tabular}
\end{Large}

\vspace*{0.8cm}
\begin{Large}
\fbox{
\begin{minipage}[t][2.5cm][t]{6cm}
Note:
\end{minipage}
}
\end{Large}

\end{titlepage}

\tableofcontents

\newpage

\section{Einleitung}
\label{sec:einleitung}
In dieser Hausarbeit stehen die Bewegungen eines Doppelpendels im Vordergrund.\\

BILD\\
Man kann es mit folgendem System gekoppelter Differentialgleichungen beschreiben.
\begin{align}
	Ml_1\ddot{\varphi}_1 + m_2l_2\ddot{\varphi}_2\cos\left(\varphi_1-\varphi_2\right) + m_2l_2\dot{\varphi}_2^2\sin\left(\varphi_1-\varphi_2\right) + Mg\sin\varphi_1 &= 0 \label{eq:bew1} \\
	m_2l_2\ddot{\varphi}_2 + m_2l_1\ddot{\varphi}_1\cos\left(\varphi_1-\varphi_2\right) - m_2l_1\dot{\varphi}_1^2\sin\left(\varphi_1-\varphi_2\right) + m_2g\sin\varphi_2&=0 \label{eq:bew2}
\end{align}
Dabei ist $M=m_1+m_2$.
Zuerst soll dieses System 2.Ordnung auf eines erster Ordnung zurückgeführt werden, um diese anschließend mittels Runge-Kutta-Verfahren 2. und 4.Ordnung numerisch zu integrieren.\\
Man fixiert $l_1=1\si{\meter}$ und $m_1=1\si{\kilo\gram}$ und variiert die Verhältnisse $l_2/l_1$ und $m_2/m_1$.
Für die Anfangsbedingungen $\varphi_1(0)=\frac{\pi}{4}$, $\dot{\varphi}_1=1 \frac{\si{rad}}{\si{\second}}$ und $\varphi_2(0)=\frac{\pi}{3}$, $\dot{\varphi}_2=1.5 \frac{\si{rad}}{\si{\second}}$ wird dann das Verhalten des Pendels untersucht.
Es soll die Bahn der beiden Massen geplottet werden sowie ihre Geschwindigkeiten in Abhängigkeit des Ortes.
Neben dem qualitativen Verhalten des Pendels sollen auch die beiden genutzten Integrationsverfahren hinsichtlich Geschwindigkeit und Genauigkeit getestet werden.

\section{System umschreiben}
Stellt man \eqref{eq:bew2} nach $\ddot{\varphi}_2$ um und setzt man dies in \eqref{eq:bew1} ein, kann diese Gleichung nach $\ddot{\varphi}_1$ aufgelöst werden. Analog stellt man \eqref{eq:bew2} nach $\ddot{\varphi}_1$ um und setzt dies in \eqref{eq:bew1} ein, um $\ddot{\varphi}_2$ zu erhalten. 
\begin{align}
	\ddot{\varphi}_1 &= -\left(m_2l_1\dot{\varphi}_1^2sc - m_2g\sin\varphi_2c + m_2l_2\dot{\varphi}_2^2s+Mg\sin\varphi_1 \right)/ \left(l_1M-l_1m_2c^2\right)\\
	\ddot{\varphi}_2 &= \left(Ml_1\dot{\varphi}_1^2s - Mg\sin\varphi_2 + m_2l_2\dot{\varphi}_2^2sc+Mg\sin\varphi_1c \right)/ \left(l_2M-l_2m_2c^2\right)
\end{align}
Hier ist $s=\sin\left(\varphi_1-\varphi_2\right)$ und $c=\cos\left(\varphi_1-\varphi_2\right)$.\\
Nun hängen die zweiten Ableitungen nicht mehr von der jeweils anderen zweiten Ableitung ab.
Substituiert man nun noch $\dot{\varphi}_1$ bsp. durch $\theta_1$ und analog $\dot{\varphi}_2$ durch $\theta_2$, sind diese 4 Gleichungen ein System erster Ordnung.
Dies kann nun numerisch gelöst werden. 

\section{Kartesische Koordinaten und Energie}
\begin{align*}
	x_1 &= l_1\sin\varphi_1\\
	y_1 &= -l_1\cos\varphi_1\\	
	x_2 &= x_1+l_2\sin\varphi_2\\
	y_2 &= y_1-l_2\cos\varphi_2\\
\end{align*}
Geschwindigkeiten
\begin{align*}
	\dot{x}_1 &= l_1\cos\varphi_1 \cdot \dot{\varphi}_1\\
	\dot{y}_1 &= l_1\sin\varphi_1 \cdot \dot{\varphi}_1\\	
	\dot{x}_2 &= \dot{x}_1 + l_2\cos\varphi_2 \cdot \dot{\varphi}_2\\
	\dot{y}_2 &= \dot{y}_1 + l_2\sin\varphi_2 \cdot \dot{\varphi}_2\\
\end{align*}
Energie
\begin{align*}
	E = E_{pot}+E_{kin} = m_1gy_1+m_2gy_2 + \frac{1}{2}m_1\left(\dot{x}_1^2+\dot{y}_1^2\right) + \frac{1}{2}m_2\left(\dot{x}_2^2+\dot{y}_2^2\right)
\end{align*}

\section{Programmaufbau}
Ich habe mich für den objektorientierten Ansatz entschieden.
So beschreibt die Klasse \textit{Doppelpendel} eben dieses. 
Sie hat folgende Attribute:
\begin{itemize}
	\item Massen und Längen	
	\item $\varphi_1$ und $\varphi_2$ sowie ihre ersten Ableitungen
	\item zugehörige kartesische Koordinaten - auch Geschwindigkeiten
	\item Energie - anfangs und aktuell
	\item Saltozähler
	\item Zeit???
\end{itemize}
Der \textit{Konstruktor} setzt $m_1=1\si{\kilo\gram}$ und $l_1=1\si{\meter}$, da wir uns hier nur auf diesen Fall beschränken.
Da alle Attribute jedoch \textit{public} sind, könnte man diese beiden auch ändern.\\
Berechnung der Trajektorie\\
Außerdem hat sie noch weitere Methoden:
\begin{itemize}
	\item $\ddot{\varphi}_1\left(\varphi_1, \dot{\varphi}_1,\varphi_2, \dot{\varphi}_2\right)$ und $\ddot{\varphi}_2\left(\varphi_1, \dot{\varphi}_1,\varphi_2, \dot{\varphi}_2\right)$
	\item Runge-Kutta 2. und 4.Ordnung	
	\item Berechnung der kartesischen Koordinaten und deren Ableitungen
	\item Berechnung der Gesamtenergie
\end{itemize}

\section{Vergleich der Integrationsverfahren}
Zwar ist Runge-Kutta 2.Ordnung mit gleicher Integrationsschrittweite schneller als der 4.Ordnung, aber auch ungenauer. 
Dies ist bei einem chaotischen System wie diesem unerwünscht.

\section{Entwicklung der Gesamtenergie}
Um die Genauigkeit des Integrationsverfahrens zu überprüfen, kann man sich die Gesamtenergie anschauen.
Diese sollte bekanntlich konstant sein.
In Abbildung ... ist die Energie gegen die Zeit aufgetragen.
Man erkennt, dass es Energiespitzen gibt. 
Die Gesamtenergie fällt jedoch in Fall A ungefähr auf das anfängliche Niveau zurück.
In Fall B klettert die Energie unaufhörlich, die Simulation ist also nur zu einem gewissen Grad brauchbar.

\section{Anhang}

\end{document}
